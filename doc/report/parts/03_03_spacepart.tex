
\subsection{Space Partitioning}

``Space Partioning'' ist einer der mehreren im Paper ``RPG - Heuristics for the
Generation of Random Polygons'' von T.Auer und M. Held~\cite{held98polygons}
beschriebenen Algorithmen zur Generierung von einfachen Polygonen.

Es handelt sich dabei um einen Divide und Conquer Algorithmus, der, wie der
Name schon suggeriert, im Divide-Schritt eine vorgegebene Punktmenge in
Partitionen mittels einem Halbebenenschnitts aufteilt, dann rekursiv Polygone
in diesen Halbebenen erzeugt und im Merge-Schritt diese Polygone in ein einfaches
gesamt Polygon zusammensetzt.

\subsubsection{Algorithmus}

Wie oben schon beschrieben ist ``Space Partitioning'' ein Divide \& Conquer Algorithmus,
der in der Betrachtung und Umsetzung die Allgemeine Lage voraussetzt. \\

\noindent
Space Partioning besteht im wesentlichen aus drei Schritten:
\begin{itemize}
  \item[1.] Die Berechnung einer Geraden $l$, die den Rand der Halbebenen repr"asentiert
            und somit die Punktmenge in zwei eindeutige Partitionen teilt
  \item[2.] Die rekursive Konstruktion eines linken und rechten Teilpolygons innerhalb
            der Partitionen
  \item[3.] Die Zusammenf"ugung dieser Teilpolygone zu einem gesamten Polygon
\end{itemize}

\noindent
Bevor die Rekursion im Space Partioning Algorithmus ausgef"uhrt werden
kann, muss diese gesondert initialisert werden, in dem wir einmalig zwei Punkte aus der
Punktmenge zuf"allig ausw"ahlen und die Partition anhand der Geraden, die durch diese
zwei gew"ahlten Punkte verl"auft, vornehmen. \\
Der eigentliche Rekursionsschritt braucht dabei diese Anfangs- und Endpunkte, um eine
neue Partitionsgerade zu konstruieren. \\
In der Implementierung wurde darauf geachtet, dass die in der Rekursion erstellten
Teilpolygone die Anfangs- bzw. Endpunkte an erster bzw. letzter Position in der
Polyline zu stehen haben - je nachdem, ob es sich um das linke oder rechte Teilpolygon
handelt -, wodurch der Merge-Schritt nur an den Grenzen der Polylines Duplikate entfernen
musste, um das Polygon zu erstellen.

% Einbauen?
%Wir k"onnen insgesamt folgende Beobachtung in einer Invariante ausdr"ucken:
%Der Schnitt der beiden Teilpolygone besitzt nur eine gemeinsame Kante und diese ist genau
%die Strecke den gegebenen Anfangs- und Endpunkte.

% TODO:
% die im Prozess der Konstruktion im Allgemeinen nicht einfach bleiben

\begin{code}[caption={Space Partioning}, mathescape=true]
choose two random points $s_f$ and $s_l$ and remove them

left, right $\leftarrow$ partition points in the left/right side of the line $l(s_f, s_l)$

leftPolygon $\leftarrow$ Space Partitioning(left, $s_f$, $s_l$)
leftPolygon $\leftarrow$ Space Partitioning(right, $s_f$, $s_l$)

polygon $\leftarrow$ merge(leftPolygon, rightPolygon)
\end{code}

\noindent

Der Rekursionsschritt l"auft analog zum Initialierungsschritt ab. Nur das hier
die Partitionierung mithilfe der Geraden $l$ (siehe Z. \ref{lst:space_line}) erfolgt.
Diese Gerade $l$ wird durch einen zuf"allig gew"ahlten Anfangspunkt auf der Strecke
$s_f$, $s_l$ und einem zuf"alligen Endpunkt aus der Punktmenge ``points'' definiert.

\begin{code}[caption={Rekursion Space Partioning}, mathescape=true, escapeinside={@}{@}]
Space Partitioning(points, $s_f$, $s_l$)

  if points only contains $s_f$ and $s_l$
    return line segment $s_f$, $s_l$

  $s$ $\leftarrow$ choose random and remove point in points
  @\label{lst:space_line}@$l$ $\leftarrow$ choose random line through $s$, which intersects the line segment $s_f$, $s_l$

  left, right $\leftarrow$ partition points in the left/right side of the line $l$

  leftPolygon $\leftarrow$ Space Partitioning(left, $s_f$, $s$)
  rightPolygon $\leftarrow$ Space Partitioning(right, $s$, $s_l$)

  polygon $\leftarrow$ merge(leftPolygon, rightPolygon)

  return polygon
\end{code}

\subsubsection{Laufzeit}

Die Laufzeitanalyse von ``Space Partitioning'' ist analog zur Laufzeitanalyse
von Quicksort. Sei $n$ die Anzahl der vorgegebenen Punktmenge.
Nach der Rekursionsbaum-Analysemethode werden in jeder Rekursionsebene 
h"ochstens $\bigO(n)$ Operationen ausgef"uhrt (Die Punktmenge wird komplett
unter den Rekursionaufrufen einer Rekursionsebene aufgeteilt).
Wodurch die Laufzeit sich nur nach der Rekursionstiefe richtet.
Bei halbwegs gleicher Aufteilung der Partitionen sind nur $\bigO(\log n)$
Rekursionsebenen n"otig, wo hingegen im worst-case $\bigO(n)$ n"otig sind. \\
Damit hat ``Space Partitioning'' eine worst-case Laufzeit von
$\bigO(n^2)$, aber in vielen F"allen eine sehr viele bessere $\bigO(n \log n)$
Laufzeit.

\subsubsection{Eigenschaften}

* Laufzeit
* Nicht Inclusive
* Implementierung: Counter Clockwise
* Schnellster implementierter Algorithmus
