\section{Statistiken}
  \subsection{Erzeugung von Daten}

    Zur Erzeugung von Daten steht ein Kommandozeileninterface zur Verf"ugung,
    welches einem alle zur Verf"ugung stehenden Parameter durch einen Aufruf
    mit --help oder --usage ausgibt. Siehe Listing~\ref{listing_usage}\\
    Die einzelnen Parameter werden im \enquote{GNU-Stil} angegeben und haben
    folgende Bedeutung:
    \begin{itemize}
      \item algorithm\\
        Dieser Parameter spezifiziert den Algorithmus der zum Generieren
        der Polygon verwendet werden soll.
      \item boundingbox\\
        Dieser Paremeter spezifiziert die l"ange der Kanten des umgebenden 
        Quadrates.
      \item number\\
        Dieser Parameter spezifiziert die Anzahl der Polygone die mit den 
        angegebenen Parametern generiert werden sollen.
      \item points\\
        Dieser Parameter spezifiziert die Anzahl an Punkten die ein
        jedes Polygon das generiert wird enthalten soll.
      \item radius\\
        Dieser Parameter betrifft nur Virmani's Velocity und spezifiziert den
        Radius des Kreises an dem die Punkte initial platziert werden.
      \item runs\\
        Dieser Parameter betrifft nur Virmani's Velocity und spezifiziert die
        Anzahl der Wiederholungen pro Durchlauf.
      \item threads\\
        Dieser Parameter spezifiziert die Anzahl der Threads auf die die 
        Generierung aufgeteilt werden soll.
      \item output\\
        Dieser Parameter spezifiziert die Zieldatei, falls keine Angegeben wurde
        wird auf \enquote{stdout} ausgegeben.
      \item no-header\\
        Dieser Parameter spezifiziert ob die Bezeichnung der Spalten mit in die
        Ausgabe geschrieben werden soll.
      \item no-statistics\\
        Dieser Parameter schreibt nur den Header in die Ausgabe.
    \end{itemize}
    Da die Angabe der Parameter je nach Art der gezielten Auswertung stark
    variieren kann empfehlen wir die Verwendung einer Scriptsprache um den
    Aufruf des Programms an die Bed"urfnisse der eigenen Untersuchung 
    anzupassen.\\
    Wir haben in unserem Fall diese Scripte f"ur die am Ende des Kapitels 
    angeh"angten Statistiken verwendet.\\
    \begin{code}[caption={Script zur Datenerzeugung},label=listing_datacreation]
#!/usr/bin/env ruby
require 'open3'

File.delete(ARGV[1]) if File.exists?(ARGV[1])
ARGV[2].to_i.times do |i|
  if i == 0
    `java -jar polygonsSWP-bin.jar --algorithm #{ARGV[0]} --threads 8 --number 10 --points #{i+4} >> #{ARGV[1]}`
  else
    Open3.popen2("java -jar polygonsSWP-bin.jar --algorithm #{ARGV[0]} --threads 8 --number 10 --points #{i+4}") do |stdin, stdout, thr|
      stdout.each do |l|
        unless l.start_with?("polygon")
          File.open(ARGV[1],"a").write(l)
        end
      end
    end
  end
end
    \end{code}\\
    Bei diesem Script k"onnen wir mithilfe des Parameters angeben welcher
    Algorithmus verwendet soll, mit dem zweiten in welche Datei geschrieben 
    werden soll und mit dem dritten bis zu welcher Punktanzahl Polygone
    generiert werden soll.
  \subsection{Visualisierung der Daten}
    Zur Visualisierung der Daten haben wir ein Tool in \enquote{Ruby} entwickelt
    mit dem es m"oglich ist die CSV-Dateien zu parsen, danach die Daten
    anhand der Punkte aufzusummieren und schlussendlich daraus einen Plott
    mit Hilfe von \enquote{gnuplot} zu erstellen.\\
    Die Angabe der Parameter f"ur dieses Tool erfolgen in YAML Syntax und
    sind wie folgt strukturiert:\\
        \begin{code}[caption={Plotkonfiguration},label=listing_plotconfiguration]
sp:
  data: space.csv
  doc_title: Space Partitioning
  diagram0:
    title: Time for Creation
    xlabel: points n
    ylabel: time in s
    style: lines
    first_value: number_of_points
    second_value: time_for_creating_polygon
  diagram1:
    title: Circumference over points
    xlabel: points n
    ylabel: time in s
    style: lines
    first_value: number_of_points
    second_value: circumference
    \end{code}\\
    Wie man sieht gibt man eine Gruppierung, die eine CSV-Datei als Datenquelle
    hat und spezifiziert auf dieser Beliebig viele Diagramme die 
    aus den Statistiken der Polygone zu erstellen sind.\\
    Der Aufruf des Programm erfolgt nun per Kommandozeilenparameter, wobei
    man sich durch --help die nun folgenden Parameter anzeigen lassen kann:\\
    \begin{code}[caption={Plottyparameter},label=listing_plottyparameter]
Usage: plotty
    -h, --help
    -c, --conf [config] 
    -t, --template [template]
    -o, --output [out]
    \end{code}\\
    Das gesamte Projekt kann man unter http://github.com/bigzed/plotty 
    herunterladen, da es zu einer eigenen Umgebung ausgearbeitet werden soll und
    eigentlich nicht teil dieses Projektes ist.
  \subsection{Diagramme}
