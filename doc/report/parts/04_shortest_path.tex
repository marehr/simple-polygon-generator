\section{Shortest Path}
	%abbildung einfügen

  Unsere Implementierung des Shortest-Path Algorithmus basiert auf dem Paper
  \enquote{Constant-Work-Space Algorithms For Geometric Problems} Asano et.
  al.~\cite{asano11shortestpath}.

  \subsection{Algorithmus}

    Der Algorithmus benutzt ein Tripel aus drei Punkten ($p$,$q_1$,$q_2$) um
    durch das Polygon zu navigieren. Es gilt, dass der Punkt $p$ ein Punkt des
    Shortest-Path ist und das Polygon hinter $q_1$, $q_2$ den Zielpunkt ($t$)
    enthält. Um dies zu erreichen wird zu Beginn das Polygon in Trapeze
    unterteilt. Vom Startpunkt aus wird ein Dreieck zu jeweils zwei
    benachbarten Eckpunkten des Trapezes aufgespannt und geprüft, ob das
    dahinter liegende Polygon den Zielpunkt enthält. Der triviale Fall, in dem
    sich der Zielpunkt im gleichen Trapez befindet wie der Startpunkt wird zu
    Beginn überprüft. Ausgehend von dem Trapez in dem der Startpunkt liegt
    wird das Polygon so reduziert, dass sich der Zielpunkt immer noch
    innerhalb des Polygons befindet. Um den nächsten Punkt des Shortest-Path
    zu finden wird die Krümmung von $p$,$q_1$,\texttt{succ}($q_1$) bzw
    $p$,$q_2$,\texttt{pred}($q_2$) betrachtet, wobei \texttt{pred} und
    \texttt{succ} jeweils den Vorgänger und Nachfolger des jeweiligen Punktes
    in der Polygonkette bezeichnen.

  \subsection{Implementierung}

    Bei der Implementierung konnten wir uns größten Teils an die Vorgaben des
    Paper halten. Allerdings stellte sich das Unterteilen des Polygons in
    Trapeze als schwieriger heraus, als zunächst angenommen. Schließlich
    entschieden wir uns, auf die Unterteilung zu verzichten und stattdessen
    einen etwas einfacheren Ansatz zu verwenden. Die Initialisierung des
    Starttripel funktioniert folgendermaßen:

    \begin{enumerate}
      \item Finde alle Eckpunkte des Polygons, die direkt vom Startpunkt aus
            sichtbar sind.
      \item Erzeuge von je zwei Eckpunkten und dem Startpunkt ein Subpolygon
            und überprüfe ob $t$ darin liegt.
      \item Falls $t$ in dem Polygon liegt ist die Startkonfiguration gefunden,
            andernfalls untersuche die nächsten zwei Punkte.
    \end{enumerate}

    Unter anderem aus diesem Grund haben wir uns vorwiegend darauf konzentriert, eine
    funktionierende Implementierung umzusetzen. Laufzeit und Speicherverbrauch
    sind daher nicht optimal und es gibt sehr viel Optimierungspotential. Daher konnten 
    wir auch nicht dem ursprünglichen Anspruch von konstantem Speicherverbauch gerecht werden.

\newpage
  \subsection{Laufzeit}

  Die Laufzeit der Implementierung lässt sich mit $O(5n^2)$ abschätzen:

  \begin{itemize}
  \item Initialisierung des Tripel: $O(2n^2)$
        \begin{itemize}
        \item Berechnung der vom Startpunkt aus sichtbaren Punkte: $O(n^2)$\\
        Um zu überprüfen, welche Punkte vom Startpunkt aus sichtbar sind
        werden alle Streckenabschnitte jeweils vom Startpunkt bis zu einem
        Eckpunkt des Polygons mit jeder Kante des Polygons geschnitten.
        \item Überprüfung der Initialisierung des Tripel: $O(n^2)$\\
        Für jeden Tupel von sichtbaren Punkten wird das Polygon reduziert
        (weitere Erklärungen unten) und anschließend geprüft ob $t$ darin liegt.
        \end{itemize}
  \item Nach der Initialisierung wird solange der Zielpunkt nicht sichtbar ist
        der nächste Schritt berechnet: $n \cdot O(3n) = O(3n^2)$
        \begin{itemize}
        \item Überprüfung, ob der Zielpunkt sichtbar ist: $O(n)$\\
        Die Strecke ($p,t$) wird mit jeder Kante des Polygons geschnitten wird.
        \item Berechnung des nächsten Schrittes: $O(2n)$
          \begin{itemize}
          \item Zuerst wird das Polygon reduziert: $O(n)$
          \item Die Krümmung der nächsten Kante des Polygons berechenen: $O(1)$
          \item Der Schnittpunkt zwischen dem Stahl (abhängig von der vorher
          berechneten Krümmung) und Polygon wird berechnet: $O(n)$
          \item Das neue Tripel wird zurückgegeben: $O(1)$
          \end{itemize}
        \end{itemize}
  \end{itemize}

  Weitere Erklärungen:

  \begin{itemize}
  \item Reduzierung des Polygons: $O(n)$\\
        Die Polygon Punktliste wird so umgestellt, dass sich ($q_1$) am Anfang der
        Liste befindet ($O(1)$). Es werden schrittweise Punkte hinzugefügt, bis
        der Punkt ($q_2$) gefunden wird ($O(n)$). Das Polygon kann jetzt mit dem
        Hinzufügen von ($q_2$) und ($p$) geschlossen werden.
  \item Schnitt von Polygon und Strahl: $O(n)$\\
        Der Strahl wird mit jeder Kante des Polygons geschnitten
  \end{itemize}

  \subsection{Speicherverbrauch}

  Der Speicherverbrauch lässt sich mit $O(n^2)$ abschätzen. Dies lässt sich zum
  einen dadruch erklären, dass wir den Algorithmus in Java implementiert haben
  und so praktisch keine Einfluss auf die Speicherverwaltung nehmen können. Zum
  anderen haben wir, um Code-Duplizierung zu vermeiden, die meisten Funktionen
  ausgelagert um sie auch in anderen Methoden benutzten zu können. Dabei haben wir
  meist keine Speicherbereiche mit übergeben, welches natürlich einen stark
  erhöhten Speicherverbauch mit sich bringt. Mit ein wenig Optimierung sollte man
  aber auch unserse Implementierung mit einem Speicherverbrauch von $O(n)$
  realisieren können. Wenn man den originalen konstanten Speicherverbrauch
  erreichen möchte sollte man die Programmiersprache C verwenden.






