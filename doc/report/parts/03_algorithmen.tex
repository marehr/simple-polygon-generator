\section{Algorithmen}
In Tabelle~\ref{algo_table} findet sich eine Auflistung der in der entstandenen Softwarelösung implementierten Algorithmen. Die meisten dieser Algorithmen bedürfen einer vorgegebenen Punktmenge, aus welcher ein simples Polygon konstruiert wird. Einige andere beginnen mit keinen oder nur wenigen vorgegebenen Punkten und erzeugen in einem gegebenen begrenzten Raum eine gewählte Anzahl weiterer Punkte. Die hier angegebene \enquote{Laufzeit} bezeichnet die theoretisch kleinste obere Schranke für das asymptotische Wachstum der Laufzeitsfunktion der Algorithmen, in Abhängigkeit der Anzahl der Punkte $n$.
\begin{table}[ht]
\begin{center}
\caption{Polygon-Algorithmen}
\begin{tabular}{lcc} 
\toprule
Algorithmus & vorg. Punktmenge & theo. Laufzeit \\
\midrule
Random Polygon Algorithm & & ?? \\
Permute \& Reject & $\checkmark$ & n/a \\
\bottomrule
\end{tabular}
\label{algo_table}
\end{center}
\end{table}
blabla