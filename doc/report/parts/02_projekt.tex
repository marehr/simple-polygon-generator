\section{Projektstruktur}

  Das Projekt \emph{Zufällige Polygone} wurde im Wintersemester 2011/2012 in
  einem Zeitraum von rund 4 Monaten durchgeführt. Im Folgenden werden wir
  zunächst die Ziele des Projekts darstellen, den zeitlichen Ablauf darlegen
  und abschließend den gewählten Entwicklungsprozess beschreiben.

  \subsection{Zielsetzung \& Einschränkungen}

    Das Hauptziel des Projekts war die Implementierung einer Reihe von
    Algorithmen, die der Erzeugung zufälliger einfacher Polygone dienen. Vom
    verantwortlichen Dozenten wurden dabei vier wissenschaftliche Artikel zur
    Verfügung gestellt, in denen derartige Algorithmen beschrieben
    werden. Darüberhinaus sollte ein vom Dozenten mitentwickelter \emph
    {Shortest-Path}-Algorithmus~\cite{asano11shortestpath} implementiert werden,
    welcher in einem gegebenen Polygon die kürzeste Verbindung zwischen zwei
    gewählten Punkten errechnet. Folgende weitergehenden Anforderungen an die zu
    entwickelnde Software waren gegeben:

    \begin{itemize}
      \item Massenweise (\emph{batch}) Erzeugung von zufälligen Polygonen 
            per Kommandozeile
      \item Export eines oder vieler Polygone als CSV-Datei 
            (\emph{comma-separated value})
      \item Grafische Oberfläche
      \begin{itemize}
        \item Auswahl und Konfiguration des Algorithmus
        \item Visualisierung der Funktionsweise des Algorithmus
        \item Auswahl der Punkte für den Shortest-Path-Algorithmus
        \item Visualisierung der Funktionsweise des Shortest-Path-Algorithmus
      \end{itemize}
      \item Statistische Analyse der Algorithmen hinsichtlich ihrer Laufzeit 
            und den Eigenschaften der erzeugten Polygone
    \end{itemize}

    Als Programmiersprache wählten wir \emph{Java}, da alle Projektmitglieder
    über Erfahrung in der Java-Programmierung verfügten und die Sprache sowohl
    die gewünschte Performanz als auch eine ausreichend umfangreiche
    Standardbibliothek bot. Aufgrund des begrenzten zeitlichen Rahmens des
    Projekts mussten jedoch einige Einschränkungen in Bezug auf die
    algorithmische Robustheit der entwickelten Software getroffen werden.
    Zunächst ist hier die Entscheidung für die Verwendung von Gleitkomma- und
    gegen die (ausschließliche) Verwendung von Ganzzahlen zu nennen. Zwar
    ermöglichen Gleitkommazahlen die einfache und CPU-gestützte
    Divisionsoperation, jedoch kann es durch die begrenzte Auflösung der
    Gleitkommazahlen (bspw. 64 Bit bei Variablen des Typs \emph{double}) zu
    Rundungsungenauigkeiten kommen.

    Unter Absprache mit dem verantwortlichen Dozenten verzichteten wir auf einen
    \emph{General Position}-Test (im Deutschen manchmal \enquote{allgemeine
    Lage}) vor Ausführung der Algorithmen. Eine Menge von Punkten in der Ebene
    befindet sich in \emph{general position}, wenn

    \begin{enumerate}
      \item keine zwei Punkte identisch sind,
      \item keine drei Punkte auf einer Geraden liegen und
      \item keine 4 Punkte auf einem Kreis liegen.
    \end{enumerate}

    Manche Polygon-Algorithmen versagen, wenn eine dieser Eigenschaften,
    insbesondere 2., für die gegebene Punktmenge nicht erfüllt ist, so dass dies eine bedeutende Einschrenkung darstellt. Zwar lässt
    sich eine Punktmenge leicht auf GP überprüfen, jedoch ist dieser Test vor
    allem für große Punktmengen sehr rechenintensiv. Der von uns vorübergehend
    verwendete naive Algorithmus zum GP-Test wies ein asymptotisches Wachstum
    von $\bigO\left(n^3\right)$ auf, sodass die Laufzeit des GP-Tests bisweilen
    die Laufzeit des Algorithmus weit übertraf. Als Konsequenz hieraus
    verzichteten wir auf einen GP-Test, sodass eine nicht in GP befindlichen
    Punktmenge zu undefinierte Ergebnissen des Programms führen kann. Mit der
    Verwendung von 64-bit \emph{double}-Variablen zur Darstellung der $x$- und
    $y$-Koordinaten ist eine solche Situation jedoch äußerst unwahrscheinlich.

  \subsection{Zeitplan}

    Wir präsentieren hier den Zeitplan des Projekts in Form eines GANTT-
    Diagramms, welches auch während der Projektarbeit Verwendung fand. Da das
    Diagramm laufend an den aktuellen Entwicklungsstand angepasst wurde,
    entspricht es nun nahezu exakt dem tatsächlichen zeitlichen Ablauf der
    Entwicklung. Die Verzögerungen, die während des Projekts auftraten, beliefen
    sich jedoch nie als mehr auf eine Woche.

    \begin{figure}[h]
    \begin{center}
    \bfseries\fontsize{5}{5}\selectfont
    \newpsstyle{TaskStyleA}{fillstyle=solid,fillcolor=red}
    \begin{PstGanttChart}[% 
        yunit=1.0,
        xunit=0.68,
        ChartUnitIntervalName=KW,
        ChartUnitBasicIntervalName=KW,
    %    ChartModulo,
        ChartModuloValue=52,
        TaskUnitIntervalValue=14,
        TaskUnitType=KW,
        ChartStartInterval=43,
        ChartShowIntervals]{15}{15}
    \psset{TaskStyle=TaskStyleA}
    \PstGanttTask[TaskInsideLabel={PolygonGenerator interface}]{0}{4}
    \PstGanttTask[TaskInsideLabel={Data structures scratch}]{0}{4}
    \PstGanttTask[TaskInsideLabel={GUI scratch}]{0}{4}
    \PstGanttTask[TaskInsideLabel={Data structures done}]{1}{3}
    \PstGanttTask[TaskInsideLabel={First polygon algorithms}]{1}{4}
    \PstGanttTask[TaskInsideLabel={ShortestPathGenerator interface}]{1}{4}
    \PstGanttTask[TaskInsideLabel={All polygon algorithms}]{3}{4}
    \PstGanttTask[TaskInsideLabel={Shortest path algorithm}]{3}{4}
    \PstGanttTask[TaskInsideLabel={GUI}]{3}{4}
    \PstGanttTask[TaskInsideLabel={History data structure}]{6}{4}
    \PstGanttTask[TaskInsideLabel={Step-by-step visualisation}]{6}{4}
    \PstGanttTask[TaskInsideLabel={Statistics backend}]{6}{4}
    \PstGanttTask[TaskInsideLabel={Improved GUI}]{10}{3}
    \PstGanttTask[TaskInsideLabel={Statistics visualisation}]{10}{3}
    \PstGanttTask[TaskInsideLabel={Cleanup for release}]{12}{3}
    \end{PstGanttChart}
    \end{center}
    \label{fig:gantt}
    \end{figure}

  \subsection{Entwicklungsprozess}

    Der von uns gewählte Entwicklungsprozess folgte keinem fest definierten
    Modell (bspw. SCRUM), beinhaltete jedoch einige prozessunterstützende
    Maßnahmen, die vor allem der kontinuierlichen Kommunikation innerhalb des
    Teams dienen sollten. Die folgenden Prozesselemente stellten sich als
    hilfreich heraus und wurden während der Projektarbeit konsequent
    beibehalten:

    \begin{itemize}
      \item Iterative Herangehensweise, stets lauffähige Version im Repository.
      \item Wöchentliche Team-Meetings zur Besprechung des aktuellen Stands, 
            der in der vergangenen Woche erledigten Arbeit sowie der für die 
            folgende Woche geplanten Arbeitspakete. Schriftliche Protokolle 
            dienten der Information abwesender Teammitglieder.
      \item Arbeitspakete werden stets in 2er-Teams bearbeitet. Die Teampartner 
            stellen sich gegenseitig erarbeiteteAlgorithmen vor und Reviewen 
            geschrieben Code.
      \item Testen der zentralen Komponenten.
      \item Unregelmäßig stattfindende \emph{coding sessions}.
    \end{itemize}

    Insbesondere der letztgenannte Punkt entwickelte sich im Laufe des Projekts
    zu einer wichtigen Maßnahme, um sowohl Verzögerungen im Zeitplan gering zu
    halten, als auch das Wissen über den geschriebenen Code möglichst
    gleichmäßig auf die Teammitglieder zu verteilen. Die \emph{coding sessions}
    wurden von allen Teammitgliedern als überaus produktiv empfunden, oftmals
    konnten an einem Tag eine ganze Reihe Arbeitspakete abgeschlossen werden.
    Zudem zeigten sich Fehler im Code oft erst, während mehrere, vorher
    unbeteiligte Teammitglieder den Code der anderen ausgiebig testeten.

    Entscheidungen im Team wurden weitestgehend per Konsens geschlossen. Für den
    seltenen Fall einer Unstimmigkeit wurde zuvor Jannis Ihrig zum Projektleiter
    bestimmt. Der Projektleiter vertrat das Projekt zudem als Ansprechpartner
    gegenüber dem Dozenten.
