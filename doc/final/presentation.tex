\documentclass[ucs,9ptb]{beamer}

\usepackage[utf8]{inputenc}
\usepackage[english]{babel}
\usepackage{graphicx}

\include{fu-beamer-template}

\titleimage{polygons}

\title[Random Polygons]{Random Polygons}
\subtitle{final presentation}
\institute[FU Berlin]{Freie Universität Berlin}
\date[06.12.2011]{6th December 2011}

\begin{document}

\begin{frame}[plain]
  \titlepage
\end{frame}


\begin{frame}{Goals}
	\begin{itemize}
	\item Testing framework for algorithms related to random polygons
	\item Generate random polygons (fair, representative)
	\item Run shortest path algorithm on polygon
	\item Statistical analysis
	\end{itemize}
\end{frame}

\begin{frame}{3. Milestone}
  \begin{itemize}
  \item shortest path (leftover from last milestone)
  \item step-by-step visualisation
  \item backend for statistics
  \item visualization for statistics
  \end{itemize}
\end{frame}

\begin{frame}{development process\\- milestones}
  \begin{itemize}
    \item 2. milestone 1 week late
    \item 3. milestone as well
    \item 1 week buffer was planned and used
  \end{itemize}
\end{frame}

\begin{frame}{development process\\- time management}
  \begin{itemize}
    \item first used long, now double
    \item therefore own framework
    \item time for extensive testing missed
    \item java as language?
    \begin{itemize}
      \item pros: portable, fast, teached at fu
      \item cons: slow development process, garbage collection
      \item next time modern language (scala, ruby ...)?
    \end{itemize}
    \item building real toolchain
  \end{itemize}
\end{frame}

\begin{frame}{development process\\- communication}
  \begin{itemize}
    \item 6 people
    \item only 1 meeting per week?
    \item next time:
    \begin{itemize}
      \item 1 for organizational stuff
      \item 1 for developement
    \end{itemize}
    \item using redmine?
    \begin{itemize}
      \item not used to extensive use of redmine
      \item more consequetly report bugs + features
    \end{itemize}
  \end{itemize}
\end{frame}

\begin{frame}{What did we accomplish?\\- shortest path}
  \begin{itemize}
    % TODO: explain algorithm
    \item original algorithm required trapezoidation
    \item 
    % TODO: runntime?
    \item runntime
    \item space complexity O(1)
  \end{itemize}
\end{frame}

\begin{frame}{What did we accomplish?\\- step-by-step visualization}
  \begin{itemize}
    \item see working algorithms live
    \item implemented by "history scenes"
    \begin{itemize}
      \item snapshot every iteration 
            all geom. objects currently used
      \item callback to gui draw
    \end{itemize}
    % TODO: besser formulieren, dass wir per gui kontrollieren koennen, welche scenen wir sehen wollen
    \item view scenes in gui
    \item or videolike playback via "play"
    \item export as svg possible
    % TODO: Marcel zeigt space partitioning und erklaehrt wie er funktioniert
  \end{itemize}
\end{frame}

\begin{frame}{What did we accomplish?\\- statistic}
  \begin{itemize}
    \item algorithmRunner runs alg. in arbitrary configuration
    \item dumps data to sqlite files
    \item plotty (ruby script)
    \begin{itemize}
      \item retrieves data from db
      \item plots via gnuplot
      \item inserts diragrams into latex file
      \item soon for you on github  ;)
    \end{itemize}
  \end{itemize}
\end{frame}

\begin{frame}{What did we accomplish?\\- analysis}
  space partitioning
  \includegraphics[scale=0.8]{spaggr/diagram0.pdf}
\end{frame}

\begin{frame}{What did we accomplish?\\- analysis}
  space partitioning
  \includegraphics[scale=0.8]{spaggr/diagram1.pdf}
\end{frame}

\begin{frame}{What did we accomplish?\\- analysis}
  2-Opt-Move
  \includegraphics[scale=0.8]{2opt/diagram1.pdf}
\end{frame}

\begin{frame}{What did we accomplish?\\- analysis}
  2-Opt-Move
  \includegraphics[scale=0.8]{2opt/diagram4.pdf}
\end{frame}

\begin{frame}{What did we accomplish?\\- analysis}
  Orouke and Virmani
  \includegraphics[scale=0.8]{virmani/diagram5.pdf}
\end{frame}

\begin{frame}{outlook}
  \begin{itemize}
    \item create real library
    \item build toolchain
    \item import of svg's
    \item develope plotty
  \end{itemize}
\end{frame}

\begin{frame}{outlook}
  \hspace{40mm}\huge{Thank you!}
\end{frame}



% was gewollt
% was geschafft
% wie entwicklungsprozess
%   welche schwierigkeiten gab es 
%   was haette man besser machen koennen
%   welche tools die wir verwendeten haben sind vorteilhaft, welche komplizieren

% statistische analyse
% ausblick

\end{document}

