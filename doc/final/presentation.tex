\documentclass[ucs,9ptb]{beamer}

\usepackage[utf8]{inputenc}
\usepackage[english]{babel}
\usepackage{graphicx}

\include{fu-beamer-template}

\titleimage{polygons}

\title[Random Polygons]{Random Polygons}
\subtitle{2. Zwischenpräsentation}
%TODO: right?
\institute[FU Berlin]{Freie Universität Berlin}
\date[06.12.2011]{6th December 2011}

\begin{document}

\begin{frame}[plain]
  \titlepage
\end{frame}


\begin{frame}{Goals}
	\begin{itemize}
	\item Testing framework for algorithms related to random polygons
	\item Generate random polygons (fair, representative)
	\item Run shortest path algorithm on polygon
	\item Statistical analysis
	\end{itemize}
\end{frame}

\begin{frame}{3. Milestone}
  \begin{itemize}
  \item shortest path (leftover from last milestone)
  \item step-by-step visualisation
  \item backend for statistics
  \item visualization for statistics
  \end{itemize}
\end{frame}

\begin{frame}{What did we accomplish?\\- shortest path}
  \paragraph{gobba} % (fold)
  \label{par:gobba}
  
  % paragraph gobba (end)
\end{frame}

\begin{frame}{What did we accomplish?\\- step-by-step visualization}
  
\end{frame}

\begin{frame}{What did we accomplish?\\- statistics}
  
\end{frame}

% was gewollt
% was geschafft
% wie entwicklungsprozess
%   welche schwierigkeiten gab es 
%   was haette man besser machen koennen
%   welche tools die wir verwendeten haben sind vorteilhaft, welche komplizieren

% statistische analyse
% ausblick

\end{document}

