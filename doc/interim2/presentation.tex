\documentclass[ucs,10pt]{beamer}

\usepackage[utf8]{inputenc}
\usepackage[english]{babel}
\usepackage{graphicx}

\include{fu-beamer-template}

\titleimage{polygons}

\title[Random Polygons]{Random Polygons}
\subtitle{2. interim}
%TODO: right?
\institute[FU Berlin]{Freie Universität Berlin}
\date[06.12.2011]{6th December 2011}

\begin{document}

\begin{frame}[plain]
  \titlepage
\end{frame}

\begin{frame}{2. Milestone}
  \begin{itemize}
  \item accomplish 6.12.-12.12.
  \item all polygon generation algorithms
  \item shortest path algorithm
  \item GUI basic version
  \end{itemize}
\end{frame}

\begin{frame}{Preperations}
  Geometry Framework:
  \begin{itemize}
    \item needed for some polygon generation algorithms
    \item existing:
    \begin{itemize}
      \item needed accuracy not given
      \item too big 
      \item partly inconsistent
      \item 
    \end{itemize}
    \item therefore small own implementation
  \end{itemize}
  \vspace*{1em}
  %TODO: remember
  2nd thing i forgot:
  \begin{itemize}
    \item sthg
  \end{itemize}
\end{frame}

\begin{frame}{Generation algorithms}
  Permute and Reject \begin{footnotesize}by Auer and Held\end{footnotesize}
  \vspace*{1em}
  \begin{itemize}\begin{small}
    \item[ ] input: size n or n points
    \item[i.] generate points, if necessary
    \item[ii.] permute points
    \item[iii.] test if polygon simple, if continue with ii.
  \end{small}\end{itemize}
  \vspace*{1em}
  \begin{itemize}\begin{small}
    \item simple to implement
    \item generates all simple polygons with uniform distribution
    \item runntime depends on polygons needed to be generated to encounter
          simple polygon (max. n=15 for fast results)
    \item not suited for practical use
  \end{small}\end{itemize}
\end{frame}

\begin{frame}{Generation algorithms}
  2opt.-Moves \begin{footnotesize}by Auer and Held\end{footnotesize}
  \vspace*{1em}
  \begin{itemize}\begin{small}
    \item[ ] input: size n or n points
    \item[i.] generate random permutation of n points
    \item[ii.] for every self intersection of two edges $ \overline{ab} $ and $ \overline{cd} $
      \begin{itemize}
        \item[a.] remove $ \overline{ab} $ and $ \overline{cd} $
        \item[b.] insert $ \overline{ac} $ and $ \overline{bd} $
      \end{itemize}
  \end{small}\end{itemize}
  \vspace*{1em}
  \begin{itemize}\begin{small}
    \item needs special treatment for polygons not in general position
    \item generates all polygons, but not with uniform distribution
  \end{small}\end{itemize}
\end{frame}

\begin{frame}{Generation algorithms}
  Space Partitioning \begin{footnotesize}by Auer and Held\end{footnotesize}
  \vspace*{1em}
  \begin{itemize}\begin{small}
    \item[ ] input: set S of n points
    \item[i.] choose points $ s_f, s_t \in S $ at random
    \item[ii.] draw line $ \overline{s_fs_t} $ $ \rightarrow $ 2 sets
    \item[iii.] recursively for all subsets S', $ s_f' $ first, $ s_t' $ last point:
      \begin{itemize}
        \item[a.] if S' only consists of $ s_f' $, $ s_t' $, return $ \overline{s_f's_t'} $
        \item[b.] choose s' from S' at random
        \item[c.] draw random line through s' intersecting $ \overline{s_f's_t'} $, thereby creating subsets S'', S''', first and last point $ s_f', s' $ and $ s', s_t' $
      \end{itemize}
  \end{small}\end{itemize}
  \vspace*{1em}
  \begin{itemize}\begin{small}
    \item generates not every possible simple polygon
    %TODO: why?
  \end{small}\end{itemize}
\end{frame}

\begin{frame}{Generation algorithms}
  Incremental Construction \& Backtracking \begin{footnotesize}by Auer and Held\end{footnotesize}
  \vspace*{1em}
  \begin{itemize}\begin{small}
    \item[ ] input: n points
    \item[i.] set of all possible edges, all unmarked, randomly choose current point s
    \item[ii.] recursively for current point:
    \begin{itemize}
      \item[a.] add next possible unmarked edge to polygon, mark all intersecting edges
      \item[b.] backtrack if no unmarked edges left and incompleted polygon
    \end{itemize}
  \end{small}\end{itemize}
  \vspace*{1em}
  \begin{itemize}\begin{small}
    \item speed depends on backtracking
    \item complex to optimize backtracking
    \item currently not suited for practical use
  \end{small}\end{itemize}
\end{frame}

\begin{frame}{Generation algorithms}
  Generating random Polygons \begin{footnotesize}by O'Rourke and Virmani\end{footnotesize}
  \vspace*{1em}
  \begin{itemize}\begin{small}
    \item[ ] input: size n, radius circle, max. speed, steps t
    \item[i.] generate random points on circle $ \rightarrow $ regular polygon
    \item[ii.] t times:
    \begin{itemize}
      \item[a.] move each vertex with random speed and direction
      \item[b.] test if move violates 2 conditions:
        \begin{itemize}
          \item polygon is simple
          \item vertices in bounding region
        \end{itemize}
      \item[c.] discard last step if violation of conditions
    \end{itemize}
  \end{small}\end{itemize}
  \vspace*{1em}
  \begin{itemize}\begin{small}
    \item still not accessible from GUI
    \item seems to generate specific class of polygons
    \item interesting for statistical analysis
  \end{small}\end{itemize}
\end{frame}

\begin{frame}{Generation algorithms (cont.)}
  \begin{center}
    \includegraphics[width=0.90\textwidth]{velocity_sample.png}
  \end{center}
\end{frame}

\begin{frame}{Generation algorithms}
  Random Polygon Algorithm \begin{footnotesize}by Dailey and Whitfield\end{footnotesize}
  \vspace*{1em}
  \begin{itemize}\begin{small}
    \item[ ] input: size n
    \item[i.] generate 3 random points $ \rightarrow $ random n-gon P size 3
    \item[ii.] randomly choose and discard edge $ \overline{ab} $
    \item[iii.] determine region P' in polygon visible from $ \overline{ab} $
    \item[iv.] randomly choose point c in P'
    \item[v.] add edges $ \overline{ac} $, $ \overline{bc} $ to polygon P
  \end{small}\end{itemize}
  \vspace*{1em}
  \begin{itemize}\begin{small}
    \item main reason for geometry framework
    \item nontrivial problem to determine visible region P'
    \item most komplex of all generaition algorithms
  \end{small}\end{itemize}
\end{frame}

\begin{frame}{Nontrivial resulting algorithmic problems}
  \begin{itemize}\begin{small}
    \item order of points representing polygon: cc-wise
    \item triangularization
    \item random point in polygon
    \item intersection line with polygon
    \item surface area of polygon
  \end{small}\end{itemize}
\end{frame}

\begin{frame}{Demonstration GUI}
  \begin{itemize}\begin{small}
    \item basic verion
    \item next important step:
      better way to pass parameters
  \end{small}\end{itemize}
\end{frame}

\begin{frame}{Milestones}
  This Milestone:
  \begin{itemize}\begin{small}
    \item 2 polygon generation algorithms missing
    \item shortest path generator missing
    \item possible in 1 week
  \end{small}\end{itemize}
  \vspace*{1em}
  Next Milestone:
    \begin{itemize}\begin{small}
      \item history objects
      \item step-by-step visualization
      \item statistic backend
    \end{small}\end{itemize}
\end{frame}

\end{document}

